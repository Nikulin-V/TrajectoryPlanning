\chapter{Синтез регулятора обратной связи по состоянию}
\label{ch:chap3}

\section{Общая структура регулятора}

Для стабилизации положения и ориентации планарного квадрокоптера применяется метод обратной связи по состоянию (state feedback). Данный подход использует информацию о всех переменных состояния системы для формирования управляющих воздействий.

Закон управления имеет вид:
\begin{equation}
    \mathbf{u} = \mathbf{u}_0 - \mathbf{K} \delta\mathbf{x}
    \label{eq:state_feedback}
\end{equation}

где $\mathbf{u}_0 = \begin{bmatrix} T_0 & 0 \end{bmatrix}^T = \begin{bmatrix} mg & 0 \end{bmatrix}^T$ --- управляющие воздействия в точке равновесия, $\delta\mathbf{x}$ --- вектор отклонений состояния от равновесного значения, $\mathbf{K}$ --- матрица коэффициентов обратной связи размерности $2 \times 6$.

Вектор отклонений состояния определяется как:
\begin{equation}
    \delta\mathbf{x} = \begin{bmatrix} \delta x & \delta v_x & \delta z & \delta v_z & \delta \theta & \delta \omega_\theta \end{bmatrix}^T
    \label{eq:delta_state}
\end{equation}

где:
\begin{align}
    \delta x &= x - x_{ref}, \quad \delta v_x = v_x, \\
    \delta z &= z - z_{ref}, \quad \delta v_z = v_z, \\
    \delta \theta &= \theta, \quad \delta \omega_\theta = \omega_\theta
\end{align}

В развёрнутом виде закон управления записывается как:
\begin{equation}
    \begin{bmatrix} \delta T \\ \delta M \end{bmatrix} = -\begin{bmatrix} 
        k_{11} & k_{12} & k_{13} & k_{14} & k_{15} & k_{16} \\
        k_{21} & k_{22} & k_{23} & k_{24} & k_{25} & k_{26}
    \end{bmatrix} \begin{bmatrix} \delta x \\ \delta v_x \\ \delta z \\ \delta v_z \\ \delta \theta \\ \delta \omega_\theta \end{bmatrix}
    \label{eq:state_feedback_expanded}
\end{equation}

где $\delta T = T - T_0$, $\delta M = M$.

\section{Метод линейно-квадратичного регулятора (LQR)}

Для определения матрицы коэффициентов обратной связи $\mathbf{K}$ используется метод линейно-квадратичного регулятора (LQR, Linear Quadratic Regulator). Данный метод обеспечивает оптимальное управление в смысле минимизации квадратичного функционала качества:

\begin{equation}
    J = \int_0^\infty \left( \delta\mathbf{x}^T \mathbf{Q} \delta\mathbf{x} + \mathbf{u}^T \mathbf{R} \mathbf{u} \right) dt
    \label{eq:lqr_cost}
\end{equation}

где $\mathbf{Q}$ --- положительно полуопределённая матрица весов для переменных состояния размерности $6 \times 6$, $\mathbf{R}$ --- положительно определённая матрица весов для управляющих воздействий размерности $2 \times 2$.

Матрица $\mathbf{Q}$ определяет относительную важность каждой переменной состояния. Большие значения диагональных элементов соответствуют более строгим требованиям к точности стабилизации соответствующих переменных. Матрица $\mathbf{R}$ ограничивает амплитуду управляющих воздействий, предотвращая чрезмерные значения тяги и момента.

\section{Выбор весовых матриц}

Выбор матриц $\mathbf{Q}$ и $\mathbf{R}$ определяет компромисс между качеством стабилизации и энергозатратами. Выбранные весовые матрицы:

\begin{equation}
    \mathbf{Q} = \text{diag}([100, 10, 100, 10, 50, 5]), \quad \mathbf{R} = \text{diag}([1, 0.01])
    \label{eq:Q_matrix}
\end{equation}

Большие веса для позиционных координат ($q_x = q_z = 100$) обеспечивают точную стабилизацию, средние веса для скоростей ($q_{vx} = q_{vz} = 10$) --- плавность движения. Высокий вес для угла наклона ($q_\theta = 50$) обеспечивает стабилизацию ориентации.

\section{Решение уравнения Риккати}

Матрица обратной связи $\mathbf{K}$ определяется через решение алгебраического уравнения Риккати (ARE, Algebraic Riccati Equation):

\begin{equation}
    \mathbf{A}^T \mathbf{P} + \mathbf{P} \mathbf{A} - \mathbf{P} \mathbf{B} \mathbf{R}^{-1} \mathbf{B}^T \mathbf{P} + \mathbf{Q} = \mathbf{0}
    \label{eq:riccati}
\end{equation}

где $\mathbf{P}$ --- симметричная положительно определённая матрица решения уравнения Риккати.

После нахождения матрицы $\mathbf{P}$, матрица коэффициентов обратной связи вычисляется по формуле:

\begin{equation}
    \mathbf{K} = \mathbf{R}^{-1} \mathbf{B}^T \mathbf{P}
    \label{eq:K_matrix}
\end{equation}

Для рассматриваемой системы (матрицы $\mathbf{A}$ и $\mathbf{B}$ из раздела~\ref{ch:chap2}) решение уравнения Риккати даёт следующую матрицу коэффициентов обратной связи:

\begin{equation}
    \mathbf{K} = \begin{bmatrix}
        0 & 0 & 10.0 & 5.48 & 0 & 0 \\
        -10.0 & -7.00 & 0 & 0 & 19.15 & 2.32
    \end{bmatrix}
    \label{eq:K_final}
\end{equation}

Особенностью полученной матрицы является то, что тяга $T$ управляется преимущественно по переменным высоты ($z$ и $v_z$), а момент $M$ --- по переменным горизонтального положения и ориентации ($x$, $v_x$, $\theta$, $\omega_\theta$). Это соответствует физической интерпретации: тяга непосредственно влияет на вертикальное движение, а момент --- на ориентацию, которая в свою очередь влияет на горизонтальное движение.


\section{Анализ замкнутой системы}

С учётом закона управления~\eqref{eq:state_feedback} и линеаризованной модели системы, динамика замкнутой системы описывается уравнением:

\begin{equation}
    \dot{\delta\mathbf{x}} = (\mathbf{A} - \mathbf{B}\mathbf{K}) \delta\mathbf{x}
    \label{eq:closed_loop}
\end{equation}

Собственные значения матрицы $\mathbf{A} - \mathbf{B}\mathbf{K}$ определяют динамику замкнутой системы и должны находиться в левой полуплоскости комплексной плоскости для обеспечения устойчивости.

Вычисление собственных значений показывает, что все они имеют отрицательные вещественные части:

\begin{equation}
    \lambda_1 = -223.58, \quad \lambda_2, \lambda_3 = -2.63 \pm 2.63j, \quad \lambda_4 = -3.16, \quad \lambda_5, \lambda_6 = -2.74 \pm 1.58j
    \label{eq:eigenvalues}
\end{equation}

Все собственные значения имеют отрицательные вещественные части, что гарантирует асимптотическую устойчивость замкнутой системы. Наличие одного вещественного собственного значения с большой по модулю отрицательной частью ($\lambda_1 = -223.58$) связано с быстрой стабилизацией высоты. Остальные собственные значения соответствуют более медленным режимам стабилизации горизонтального положения и ориентации с временем установления около $1.0-1.5$ с.

\section{Ограничения на управляющие воздействия}

В реальной системе управляющие воздействия ограничены физическими возможностями двигателей. При реализации регулятора применяется насыщение (saturation) к управляющим сигналам:

\begin{align}
    T &= \max(0, \min(T_{max}, T_0 + \delta T)), \quad T_{max} = 2.0 \cdot mg \\
    M &= \max(-M_{max}, \min(M_{max}, \delta M)), \quad M_{max} = 0.8 \text{ Н$\cdot$м}
    \label{eq:saturation}
\end{align}

Ограничения учитывают, что тяга не может быть отрицательной (роторы не могут создавать отрицательную тягу) и что максимальные значения определяются характеристиками двигателей.

Для улучшения качества управления при наличии ограничений может применяться модификация закона управления с учётом предсказания насыщения (anti-windup), однако для рассматриваемой системы стандартный LQR-регулятор демонстрирует приемлемые характеристики без дополнительных модификаций.

\section{Применение к нелинейной системе}

Полученный регулятор синтезирован для линеаризованной модели системы в окрестности точки равновесия. Однако он применяется к исходной нелинейной системе~\eqref{eq:full_model}:

\begin{equation}
    \mathbf{u} = \mathbf{u}_0 - \mathbf{K} \delta\mathbf{x}
    \label{eq:nonlinear_control}
\end{equation}

где отклонения состояния вычисляются относительно заданных опорных значений $x_{ref}$ и $z_{ref}$.

Такой подход обеспечивает локальную асимптотическую устойчивость в окрестности точки равновесия. Для больших отклонений от равновесного состояния качество управления может ухудшиться, однако при разумных значениях начальных условий и опорных значений система демонстрирует устойчивое поведение.

\section{Показатели качества регулирования}

При оценке качества управления используются время переходного процесса $t_s$, перерегулирование $\sigma$ и статическая ошибка $e_{ss}$. LQR-регулятор обеспечивает время переходного процесса около $1.5$--$2.5$ с и перерегулирование не более $10$--$15\%$ для оптимально настроенных весовых матриц.

Метод обеспечивает оптимальность в смысле квадратичного функционала, единую структуру управления и гарантированную устойчивость. Основным недостатком является требование полного измерения состояния.
