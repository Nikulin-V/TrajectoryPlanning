\chapter{Результаты моделирования}
\label{ch:chap4}

\section{Параметры симуляции}

Моделирование проводилось с использованием нелинейной модели планарного квадрокоптера и синтезированного регулятора обратной связи по состоянию (LQR).
Параметры модели и симуляции приведены в таблице~\ref{tab:sim_params}.

\begin{table}[h]
    \centering
    \caption{Параметры модели и симуляции}
    \label{tab:sim_params}
    \begin{tabular}{|l|c|c|}
        \hline
        \textbf{Параметр} & \textbf{Обозначение} & \textbf{Значение} \\
        \hline
        Масса & $m$ & 1.0 кг \\
        Момент инерции & $J$ & 0.01 кг$\cdot$м$^2$ \\
        Ускорение свободного падения & $g$ & 9.81 м/с$^2$ \\
        Шаг интегрирования & $dt$ & 0.01 с \\
        Время симуляции & $T$ & 10--15 с \\
        \hline
    \end{tabular}
\end{table}

Весовые матрицы LQR-регулятора:
\begin{itemize}
    \item $\mathbf{Q} = \text{diag}([100, 10, 100, 10, 50, 5])$;
    \item $\mathbf{R} = \text{diag}([1, 0.01])$.
\end{itemize}

Ограничения на управляющие воздействия:
\begin{itemize}
    \item Максимальная тяга: $T_{max} = 2.0 \cdot mg = 19.62$ Н;
    \item Максимальный момент: $|M|_{max} = 0.8$ Н$\cdot$м.
\end{itemize}

\section{Сценарии моделирования}

Для проверки эффективности синтезированного регулятора были рассмотрены четыре сценария с различными начальными условиями и целевыми состояниями, представленные в таблице~\ref{tab:scenarios}.

\begin{table}[h]
    \centering
    \caption{Сценарии моделирования}
    \label{tab:scenarios}
    \begin{tabular}{|c|l|c|}
        \hline
        \textbf{№} & \textbf{Описание} & \textbf{Целевое состояние} \\
        \hline
        1 & Стабилизация высоты & $z=1.5$ м \\
        2 & Стабилизация ориентации & $\theta=0$ \\
        3 & Перемещение по горизонтали & $x=2.0$ м \\
        4 & Комбинированное движение & $x=2.0$ м, $z=1.5$ м \\
        \hline
    \end{tabular}
\end{table}

\section{Показатели качества}

Для оценки качества стабилизации использовались следующие показатели:
\begin{itemize}
    \item Время переходного процесса $t_s$ --- время достижения установившегося режима (критерий 5\% коридора);
    \item Перерегулирование $\sigma$ --- максимальное отклонение от целевого значения, выраженное в процентах;
    \item Статическая ошибка $e_{ss}$ --- установившееся отклонение от целевого значения.
\end{itemize}

Результаты для различных сценариев приведены в таблице~\ref{tab:quality}.

\begin{table}[h]
    \centering
    \caption{Показатели качества стабилизации}
    \label{tab:quality}
    \begin{tabular}{|c|c|c|c|}
        \hline
        \textbf{Переменная} & $t_s$, с & $\sigma$, \% & $e_{ss}$ \\
        \hline
        $x$ & 1.2--1.5 & $< 15$ & $< 0.01$ м \\
        $z$ & 1.8--2.0 & $< 10$ & $< 0.01$ м \\
        $\theta$ & 1.5--2.0 & $< 20$ & $< 0.001$ рад \\
        \hline
    \end{tabular}
\end{table}

\section{Анализ результатов}

Результаты моделирования показывают, что синтезированный LQR-регулятор обеспечивает быстрое время переходного процесса ($t_s \sim 1.2$--2.0 с), небольшое перерегулирование ($\sigma < 20\%$) и высокую точность ($e_{ss} < 0.01$ м для координат, $< 0.001$ рад для угла).

Все собственные значения замкнутой системы имеют отрицательные вещественные части, что гарантирует асимптотическую устойчивость. Единая структура управления обеспечивает координацию между различными контурами управления.

\section{Моделирование свободного движения}

Для анализа динамических свойств системы было проведено моделирование свободного (неуправляемого) движения. При отсутствии управления квадрокоптер падает под действием силы тяжести, при этом угол наклона остаётся неизменным при нулевой угловой скорости, что приводит к горизонтальному дрейфу.

Результаты моделирования свободного движения показаны на рисунке~\ref{fig:free_motion}. Видно, что при начальном угле наклона $\theta_0 = 0.1$ рад ($\approx 5.7°$) аппарат начинает двигаться в горизонтальном направлении, одновременно падая под действием гравитации. Время падения с высоты $z_0 = 1.0$ м до поверхности ($z = 0$) составляет около $3.0$ с, что соответствует физике системы.

Данное моделирование подтверждает, что без управления система неустойчива и не может самостоятельно поддерживать заданное положение, что подтверждает необходимость использования регулятора.

\begin{figure}[h]
    \centering
    \includegraphics[width=0.9\textwidth]{media/plots/free_motion.png}
    \caption{Свободное движение планарного квадрокоптера при начальных условиях $x_0=0$, $z_0=1.0$ м, $\theta_0=0.1$ рад}
    \label{fig:free_motion}
\end{figure}

\section{Графики переходных процессов}

На рисунках~\ref{fig:x_position}--\ref{fig:moment} представлены результаты моделирования переходных процессов для переменных состояния и управляющих воздействий. Графики показывают эволюцию переменных во времени при стабилизации к заданному состоянию.

\begin{figure}[h]
    \centering
    \includegraphics[width=0.8\textwidth]{media/plots/x_position}
    \caption{Переходный процесс горизонтального положения $x(t)$}
    \label{fig:x_position}
\end{figure}

\begin{figure}[h]
    \centering
    \includegraphics[width=0.8\textwidth]{media/plots/z_position}
    \caption{Переходный процесс высоты $z(t)$}
    \label{fig:z_position}
\end{figure}

\begin{figure}[h]
    \centering
    \includegraphics[width=0.8\textwidth]{media/plots/theta_orientation}
    \caption{Переходный процесс ориентации $\theta(t)$}
    \label{fig:theta_orientation}
\end{figure}

\begin{figure}[h]
    \centering
    \includegraphics[width=0.8\textwidth]{media/plots/thrust_T}
    \caption{Управляющее воздействие: тяга $T(t)$}
    \label{fig:thrust}
\end{figure}

\begin{figure}[h]
    \centering
    \includegraphics[width=0.8\textwidth]{media/plots/moment_M}
    \caption{Управляющее воздействие: момент $M(t)$}
    \label{fig:moment}
\end{figure}

\section{Анализ управляющих воздействий}

Управляющие воздействия остаются в допустимых пределах: максимальная тяга не превышает $1.5 \cdot mg$, момент ограничен $|M| \leq 0.8$ Н$\cdot$м. Управляющие воздействия формируются как линейная комбинация всех переменных состояния, что обеспечивает согласованное управление.

\section{Моделирование при различных начальных условиях}

Для оценки робастности синтезированного регулятора было проведено моделирование при различных начальных условиях, отличающихся от точки равновесия. Рассмотрены три сценария:

\begin{enumerate}
    \item Большое начальное отклонение по углу ($\theta_0 = 0.3$ рад, $\approx 17°$);
    \item Большое начальное отклонение по горизонтали ($x_0 = 1.0$ м);
    \item Большое начальное отклонение по высоте ($z_0 = 0.5$ м).
\end{enumerate}

Результаты моделирования показаны на рисунках~\ref{fig:initial_large_theta}--\ref{fig:initial_large_z_dev}. Во всех случаях система успешно стабилизируется к заданному состоянию ($x_{ref} = 0$, $z_{ref} = 1.0$ м, $\theta_{ref} = 0$), демонстрируя способность регулятора компенсировать большие начальные отклонения.

\begin{figure}[h]
    \centering
    \includegraphics[width=0.8\textwidth]{media/plots/disturbance_initial_large_theta.png}
    \caption{Стабилизация при большом начальном угле наклона ($\theta_0 = 0.3$ рад)}
    \label{fig:initial_large_theta}
\end{figure}

\begin{figure}[h]
    \centering
    \includegraphics[width=0.8\textwidth]{media/plots/disturbance_initial_large_x_dev.png}
    \caption{Стабилизация при большом начальном отклонении по горизонтали ($x_0 = 1.0$ м)}
    \label{fig:initial_large_x_dev}
\end{figure}

\begin{figure}[h]
    \centering
    \includegraphics[width=0.8\textwidth]{media/plots/disturbance_initial_large_z_dev.png}
    \caption{Стабилизация при большом начальном отклонении по высоте ($z_0 = 0.5$ м)}
    \label{fig:initial_large_z_dev}
\end{figure}

\section{Моделирование при внешних возмущениях}

Для оценки способности регулятора компенсировать внешние возмущения было проведено моделирование с применением возмущений различного типа в момент времени $t = 2.0$ с:

\begin{enumerate}
    \item \textbf{Возмущение по тяге}: импульсное увеличение тяги на $\Delta T = 2.0$ Н;
    \item \textbf{Ветровое возмущение}: горизонтальная сила, эквивалентная скорости ветра $v_w = 1.0$ м/с в течение $0.5$ с.
\end{enumerate}

Результаты моделирования показаны на рисунках~\ref{fig:dist_thrust} и~\ref{fig:dist_wind}. Во всех случаях регулятор успешно компенсирует возмущения и возвращает систему к заданному состоянию. Время восстановления после возмущения составляет около $1.5$--$2.0$ с, что соответствует динамике замкнутой системы.

\begin{figure}[h]
    \centering
    \includegraphics[width=0.8\textwidth]{media/plots/disturbance_thrust.png}
    \caption{Реакция системы на импульсное возмущение по тяге ($\Delta T = 2.0$ Н при $t = 2.0$ с)}
    \label{fig:dist_thrust}
\end{figure}

\begin{figure}[h]
    \centering
    \includegraphics[width=0.8\textwidth]{media/plots/disturbance_wind.png}
    \caption{Реакция системы на ветровое возмущение ($v_w = 1.0$ м/с при $t = 2.0$--$2.5$ с)}
    \label{fig:dist_wind}
\end{figure}
