\chapter{Построение математической модели}
\label{ch:chap1}


\section{Система координат и обозначения}

Рассмотрим планарный квадрокоптер, движущийся в вертикальной плоскости $x$--$z$, изображённый на рисунке~\ref{fig:planar_quad}.
Планарный квадрокоптер представляет собой упрощённую модель полного квадрокоптера, в которой движение ограничено одной вертикальной плоскостью, что позволяет существенно упростить анализ динамики системы при сохранении основных физических особенностей.

\begin{figure}[h!]
    \centering
    \includegraphics[width=0.6\textwidth]{media/xyz}
    \caption{Схема квадрокоптера в пространстве и система координат}
    \label{fig:planar_quad}
\end{figure}


В работе используются две системы координат:
\begin{itemize}
    \item \textbf{Инерциальная система координат} $O_{xyz}$ --- неподвижная система координат, связанная с землёй.
    Ось $X$ направлена горизонтально вперёд, ось $Z$ направлена вертикально вверх, ось $Y$ перпендикулярна плоскости движения (на рисунке не показана, так как движение происходит в плоскости $x$--$z$).
    \item \textbf{Связанная система координат} $O_{x_b y_b z_b}$ --- система координат, жёстко связанная с корпусом квадрокоптера.
    Ось $Z_b$ направлена вверх вдоль нормали к плоскости роторов, ось $X_b$ направлена вперёд вдоль продольной оси квадрокоптера.
\end{itemize}

Основными координатами квадрокоптера являются:
\begin{itemize}
    \item $x$ --- горизонтальное положение центра масс в инерциальной системе координат [м];
    \item $z$ --- высота центра масс над уровнем земли [м];
    \item $\theta$ --- угол наклона корпуса (тангаж) [рад], определяющий ориентацию связанной системы координат относительно инерциальной.
\end{itemize}

Угол $\theta$ положителен при наклоне корпуса вперёд (нос квадрокоптера опускается).
Горизонтальное перемещение возникает только за счёт изменения тангажа, поэтому крен $\phi$ (вращение вокруг продольной оси $X_b$) и рыскание $\psi$ (вращение вокруг вертикальной оси $Z$) не учитываются в модели, так как в плоскости $x$--$z$ они не влияют на динамику квадрокоптера.

\section{Уравнения поступательного движения}

Поступательное движение центра масс планарного квадрокоптера описывается законами Ньютона для материальной точки
На центр масс действуют следующие силы:
\begin{itemize}
    \item Сила тяжести $\vec{F}_g = -mg \vec{e}_z$, направленная вертикально вниз;
    \item Суммарная сила тяги роторов $\vec{T}$, направление которой зависит от ориентации корпуса;
    \item Аэродинамические силы (сопротивление воздуха, подъёмная сила), которые в первом приближении для простоты моделирования пренебрежимо малы при малых скоростях движения.
\end{itemize}

\subsection{Модель силы тяги роторов}

Каждый из четырёх роторов квадрокоптера создаёт силу тяги, пропорциональную квадрату угловой скорости вращения:
\begin{equation}
    T_i = k_f \Omega_i^2, \quad i = 1, 2, 3, 4
    \label{eq:thrust_single}
\end{equation}
где $T_i$ --- тяга $i$-го ротора [Н], $k_f$ --- коэффициент тяги [Н$\cdot$с$^2$/рад$^2$], $\Omega_i$ --- угловая скорость $i$-го ротора [рад/с].

Суммарная сила тяги всех роторов:
\begin{equation}
    T = \sum_{i=1}^{4} T_i = k_f \sum_{i=1}^{4} \Omega_i^2
    \label{eq:total_thrust}
\end{equation}

В связанной системе координат сила тяги направлена вдоль оси $Z_b$ (нормаль к плоскости роторов).
В инерциальной системе координат направление этой силы определяется углом наклона (тангажа) $\theta$ через матрицу поворота:
\begin{equation}
    \begin{bmatrix}
        T_x \\ T_z
    \end{bmatrix} = \begin{bmatrix} -\sin\theta \\ \cos\theta
    \end{bmatrix} T
    \label{eq:thrust_projection}
\end{equation}

Таким образом, проекции суммарной силы тяги $\vec{T}$ на оси инерциальной системы координат имеют вид:
\begin{align}
    T_x &= -T \sin\theta, \label{eq:Tx} \\
    T_z &= T \cos\theta. \label{eq:Tz}
\end{align}

Знак «минус» в выражении для $T_x$ обусловлен тем, что положительный угол $\theta$ (наклон вперёд) создаёт силу, направленную в отрицательном направлении оси $X$, что соответствует физической интуиции: для движения вперёд квадрокоптер должен наклониться вперёд.

\subsection{Уравнения динамики}

Применяя второй закон Ньютона к центру масс квадрокоптера, получаем уравнения движения в проекциях на оси инерциальной системы координат:
\begin{align}
    m \ddot{x} &= T_x = -T \sin\theta, \label{eq:trans_x} \\
    m \ddot{z} &= T_z - mg = T \cos\theta - mg. \label{eq:trans_z}
\end{align}

Уравнения \eqref{eq:trans_x} и \eqref{eq:trans_z} являются нелинейными и связывают поступательное движение аппарата с его ориентацией $\theta$ и управляющим воздействием $T$.

\textbf{Ключевая особенность планарного квадрокоптера}: из уравнения~\eqref{eq:trans_x} видно, что прямое управление горизонтальной координатой $x$ отсутствует.
Горизонтальное движение возникает косвенно, только при наличии ненулевого угла $\theta$, который создаёт горизонтальную проекцию силы тяги.
Это отражает неполноприводность системы: при двух управляющих воздействиях ($T$ и момент $M$, управляющий $\theta$) необходимо стабилизировать три степени свободы ($x$, $z$, $\theta$).

\subsection{Кинематические соотношения}

Кинематические соотношения связывают скорости центра масс с его координатами:
\begin{align}
    \dot{x} &= v_x, \label{eq:kin_x} \\
    \dot{z} &= v_z, \label{eq:kin_z}
\end{align}
где $v_x$ и $v_z$ --- проекции вектора скорости центра масс на оси инерциальной системы координат [м/с].

\subsection{Система уравнений поступательного движения}

Объединяя уравнения динамики~\eqref{eq:trans_x},~\eqref{eq:trans_z} и кинематики~\eqref{eq:kin_x},~\eqref{eq:kin_z}, получаем полную систему уравнений поступательного движения планарного квадрокоптера в форме Коши:
\begin{subequations}
    \label{eq:translational_dynamics}
    \begin{align}
        \dot{x} &= v_x, \\
        \dot{v_x} &= -\frac{1}{m} T \sin\theta, \\
        \dot{z} &= v_z, \\
        \dot{v_z} &= \frac{1}{m} T \cos\theta - g.
    \end{align}
\end{subequations}

Система~\eqref{eq:translational_dynamics} описывает эволюцию вектора состояния
$\begin{bmatrix}
    x & v_x & z & v_z
\end{bmatrix}^T$
в зависимости от управляющих воздействий $T$ и $\theta$ (где угол $\theta$, в свою очередь, управляется моментом $M$).


\section{Уравнения вращательной динамики}

Вращательное движение корпуса квадрокоптера относительно центра масс описывается уравнением динамики вращательного движения твёрдого тела вокруг неподвижной оси.

\subsection{Управляющий момент}

Управляющий момент $M$ (или $u_2$) создаётся за счёт разности тяг роторов, расположенных на противоположных сторонах квадрокоптера.
Для планарного квадрокоптера, движущегося в плоскости $x$--$z$, момент создаётся разностью тяг передних и задних роторов:
\begin{equation}
    M = L (T_2 + T_4 - T_1 - T_3)
    \label{eq:moment_control}
\end{equation}
где $L$ --- расстояние от центра масс до оси ротора [м], $T_1, T_2, T_3, T_4$ --- тяги роторов.

При малых углах наклона и симметричной конфигурации квадрокоптера можно считать, что управляющий момент $M$ является независимым управляющим воздействием, наряду с суммарной тягой $T$.

\subsection{Уравнение динамики вращения}

Вращательное движение корпуса относительно центра масс определяется моментом инерции $J$ и управляющим моментом $M$:
\begin{equation}
    J \ddot{\theta} = M
    \label{eq:rotational_dynamics}
\end{equation}
где $\ddot{\theta}$ --- угловое ускорение корпуса [рад/с$^2$], $J$ --- момент инерции корпуса относительно оси, проходящей через центр масс перпендикулярно плоскости движения [кг$\cdot$м$^2$].

Уравнение~\eqref{eq:rotational_dynamics} получено из основного уравнения динамики вращательного движения:
\begin{equation}
    \frac{d}{dt}(J \dot{\theta}) = M
\end{equation}
при условии, что момент инерции $J$ постоянен (жесткий корпус).

\subsection{Кинематическое соотношение}

Кинематическая связь между угловым положением и угловой скоростью задаётся выражением:
\begin{equation}
    \dot{\theta} = \omega_{\theta}
    \label{eq:kin_theta}
\end{equation}
где $\omega_{\theta}$ --- угловая скорость тангажа [рад/с].

\subsection{Система уравнений вращательного движения}

Объединяя уравнения динамики~\eqref{eq:rotational_dynamics} и кинематики~\eqref{eq:kin_theta}, получаем систему уравнений вращательного движения:
\begin{subequations}
    \label{eq:rotational_system}
    \begin{align}
        \dot{\theta} &= \omega_{\theta}, \\
        \dot{\omega_{\theta}} &= \frac{1}{J} M.
    \end{align}
\end{subequations}


\section{Полная математическая модель}

Объединяя системы уравнений поступательного~\eqref{eq:translational_dynamics} и вращательного~\eqref{eq:rotational_system} движения, получаем полную математическую модель планарного квадрокоптера:
\begin{subequations}
    \label{eq:full_model}
    \begin{align}
        \dot{x} &= v_x, \\
        \dot{v_x} &= -\frac{1}{m} T \sin\theta, \\
        \dot{z} &= v_z, \\
        \dot{v_z} &= \frac{1}{m} T \cos\theta - g, \\
        \dot{\theta} &= \omega_{\theta}, \\
        \dot{\omega_{\theta}} &= \frac{1}{J} M.
    \end{align}
\end{subequations}

Система~\eqref{eq:full_model} описывает эволюцию вектора состояния:
\begin{equation}
    \mathbf{x} = \begin{bmatrix}
        x & v_x & z & v_z & \theta & \omega_{\theta}
    \end{bmatrix}^T
    \label{eq:state_vector}
\end{equation}

в зависимости от вектора управления:
\begin{equation}
    \mathbf{u} = \begin{bmatrix}
        T & M
    \end{bmatrix}^T
    \label{eq:control_vector}
\end{equation}


\section{Параметры модели}

В таблице~\ref{tab:model_params} приведены значения параметров модели планарного квадрокоптера, используемые в симуляции.

\begin{table}[h]
    \centering
    \caption{Параметры модели планарного квадрокоптера}
    \label{tab:model_params}
    \begin{tabular}{|l|c|c|c|}
        \hline
        \textbf{Параметр}            & \textbf{Обозначение} & \textbf{Значение}   & \textbf{Единица}      \\
        \hline
        Масса                        & $m$                  & 1.0                 & кг                    \\
        Момент инерции               & $J$                  & 0.01                & кг$\cdot$м$^2$        \\
        Ускорение свободного падения & $g$                  & 9.81                & м/с$^2$               \\
        Расстояние до ротора         & $L$                  & 0.25                & м                     \\
        Коэффициент тяги             & $k_f$                & $1.5 \cdot 10^{-5}$ & Н$\cdot$с$^2$/рад$^2$ \\
        \hline
    \end{tabular}
\end{table}
