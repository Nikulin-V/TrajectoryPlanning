\chapter*{Введение}
\addcontentsline{toc}{chapter}{Введение}
\label{ch:intro}

Планарный квадрокоптер представляет собой летательный аппарат с четырьмя роторами, способный перемещаться в вертикальной плоскости.
Основными координатами такой системы являются горизонтальное положение $x$, высота $z$ и угол наклона корпуса $\theta$, который определяет наклон аппарата относительно горизонта.
Управление аппаратом осуществляется с помощью двух величин: суммарной тяги, действующей на центр масс, и управляющего момента, изменяющего угол наклона корпуса.
Особенностью планарного квадрокоптера является то, что горизонтальное движение возникает косвенно через наклон корпуса.
Это делает систему неполноприводной и создаёт дополнительные трудности при разработке алгоритмов управления, поскольку изменение одной переменной напрямую влияет на несколько аспектов движения.

В последние годы квадрокоптеры получили широкое распространение и перестали быть исключительно экспериментальными аппаратами.
Их используют в аэросъёмке, мониторинге окружающей среды, доставке грузов и даже в научных исследованиях.
Несмотря на кажущуюся простоту конструкции, управление квадрокоптером является нетривиальной задачей из-за сильной взаимосвязи движений по разным осям и высокой нелинейности системы.

В данной работе решается задача стабилизации положения и ориентации планарного квадрокоптера с использованием метода обратной связи по состоянию и линейно-квадратичного регулятора (LQR).
Этот метод позволяет оптимально выбрать управляющие воздействия, минимизируя отклонения от заданного положения и величину усилий, приложенных к системе.
Такой подход обеспечивает стабильность аппарата даже при воздействии внешних возмущений и внутренних динамических эффектов, а также даёт возможность оценить эффективность системы управления до её практической реализации.

Цель работы заключается в разработке надёжной системы стабилизации, которая позволяет удерживать квадрокоптер на заданной высоте и поддерживать правильное положение корпуса.
