\chapter*{Заключение}
\addcontentsline{toc}{chapter}{Заключение}
\label{ch:outro}

В данной работе решена задача стабилизации положения и ориентации планарного квадрокоптера методом обратной связи по состоянию с использованием линейно-квадратичного регулятора (LQR).

\textbf{Основные результаты:}

\begin{enumerate}
    \item \textbf{Математическая модель}: построена полная нелинейная модель планарного квадрокоптера, описывающая поступательное движение центра масс в вертикальной плоскости и вращательное движение корпуса. Модель учитывает связь между ориентацией и горизонтальным движением, что отражает неполноприводность системы.
    
    \item \textbf{Линеаризация и анализ}: выполнена линеаризация модели в окрестности точки равновесия. Показано, что система полностью управляема, что позволяет стабилизировать все переменные состояния с помощью двух управляющих воздействий --- суммарной тяги и управляющего момента. Все собственные значения линеаризованной системы равны нулю, что означает, что система находится на границе устойчивости и требует введения обратной связи для стабилизации.
    
    \item \textbf{Синтез регулятора}: разработан регулятор обратной связи по состоянию на основе метода линейно-квадратичного регулятора (LQR). Решение алгебраического уравнения Риккати позволило определить оптимальную матрицу коэффициентов обратной связи $\mathbf{K}$, обеспечивающую минимизацию квадратичного функционала качества. Выбраны весовые матрицы $\mathbf{Q} = \text{diag}([100, 10, 100, 10, 50, 5])$ и $\mathbf{R} = \text{diag}([1, 0.01])$, обеспечивающие баланс между качеством управления и энергозатратами.
    
    \item \textbf{Результаты моделирования}:
    \begin{itemize}
        \item Время переходного процесса: $\sim$1.2--1.5 с для горизонтального положения, $\sim$1.8--2.0 с для высоты, $\sim$1.5--2.0 с для ориентации;
        \item Перерегулирование: не превышает 15--20\% для позиционных координат;
        \item Статическая ошибка: $< 0.01$ м для координат, $< 0.001$ рад для угла наклона;
        \item Управляющие воздействия остаются в допустимых пределах;
        \item Все собственные значения замкнутой системы имеют отрицательные вещественные части, что гарантирует асимптотическую устойчивость.
    \end{itemize}
    
    \item \textbf{Преимущества метода}: LQR обеспечивает оптимальное управление, единую структуру управления и гарантированную устойчивость для линеаризованной системы.
\end{enumerate}

\textbf{Ограничения и перспективы:}

Метод обратной связи по состоянию требует наличия информации о всех переменных состояния, что может потребовать использования наблюдателя состояния или дополнительных датчиков. Синтез регулятора выполняется для линеаризованной модели, что может ограничивать рабочую область системы. Для больших отклонений от равновесного состояния качество управления может ухудшиться.

Дальнейшие направления развития работы могут включать разработку наблюдателя состояния, применение адаптивных методов управления, расширение модели для учёта аэродинамических эффектов и реализацию системы на реальном аппарате.

Полученная система управления обеспечивает эффективную стабилизацию положения и ориентации планарного квадрокоптера с высоким качеством переходных процессов и гарантированной устойчивостью, что подтверждается результатами моделирования для различных начальных условий и целевых состояний.

\endinput
